\documentclass{beamer}
\usetheme{metropolis}           % Use metropolis theme
\usepackage{minted}
\title{ICMPv6 support for libpcap}
\date{\today}
\author{Matthias Hannig, Moritz Wilhelmy}
\institute{RIPE NCC IPv6 Hackathon Copenhagen}
\begin{document}
  \maketitle
  \section{Teaching someone else's old dog new tricks}
  \begin{frame}{The Problem}
    \texttt{tcpdump -i en0 "icmp6[icmptype] = icmp6-echo"}\\
    \texttt{tcpdump: IPv6 upper-layer protocol is not supported by proto[x]}
  \end{frame}

  \begin{frame}{Enter BPF}
    Berkeley Packet Filter, a bytecode register machine that runs inside the kernel.

    Used for performing computational operations on network packets (and nowadays other data on Linux).

    Linux and FreeBSD JIT-compile the bytecode to native assembly on a variety
    of platforms for additional speed.
  \end{frame}

  \begin{frame}[fragile]{Example bytecode: IPv6 packet?}
    \begin{minted}{text}
      (000) ldh      [12]
      (001) jeq      #0x86dd          jt 2	jf 3
      (002) ret      #TRUE
      (003) ret      #0
    \end{minted}

    \begin{itemize}
      \item Load half-word (16 bit) at offset 12 (EtherType) into the accumulator
      \item Compare equality with magic value for IPv6 packet \texttt{0x86dd}
      \item If true, jump to line 2, else jump to line 3
      \item Return true if it matches, or false if it doesn't
    \end{itemize}
  \end{frame}

  \begin{frame}{What is libpcap?}
    \begin{itemize}
      \item Packet capture: OS independent engine to tap into the network stack
      \item Fallback BPF bytecode interpreter in case the kernel rejects the BPF code
      \item BPF filter code generator
    \end{itemize}
  \end{frame}

  \begin{frame}{Adding ICMPv6}

  \end{frame}

  % fehlt noch:
  % ICMPv6
  % Future developments
  % Questions?
\end{document}
